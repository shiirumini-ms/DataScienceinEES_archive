\documentclass[10,a4paperpaper,]{article}

  \title{WWF Report\newline Red Fox Population Trend in late 20th
Century}
  \author{B241110}
  \date{\today}
  


\usepackage{titlesec}
\usepackage{hyperref}
\newcommand{\pandocbounded}[1]{#1}
\newcommand{\logo}{logo.png}
\newcommand{\cover}{cover-1.png}
\newcommand{\iblue}{2b4894}
\newcommand{\igray}{d4dbde}
\usepackage{ragged2e}
\justifying

\include{defs}

\begin{document}


\renewcommand{\contentsname}{Table of Contents}

\renewcommand{\thepage}{\arabic{page}}


\maketitle
\tableofcontents
\addcontentsline{toc}{section}{Contents}
\clearpage

\section{Introduction and Summary}\label{introduction-and-summary}

\subsection{About the Species}

\emph{Vulpes vulpes} (red fox) is an omnivore widely found across the
Northern Hemisphere, thriving in temperate and boreal climate. Its
flexible diet allow them to inhabit wide range of habitats like forests,
grasslands, and urban environments. However, their adaptability also
makes the species a major carrier of \emph{Rabies lyssavirus}
(``rabies''), posing risks to wildlife populations, ecosystem dynamics,
and public health.

During 1970--2000, rabies spread across continental Europe (Delcourt et
al., 2022), but its effect on population abundance remains uncertain.

\subsection{Research Question}

This report aims to test how red fox population changed over 1970-2000
between regions affected by rabies and those free from the disease. We
expected declines where rabies occurred and stable or increasing trends
elsewhere.

\subsection{Key Findings}

Red fox populations declined by 2.3 \% per year in rabies-affected
regions and increased by 1.8 \% per year in rabies-free regions,
suggesting rabies are associated with the population decline.

\pagebreak

\section{Methods}\label{methods}

\subsection{Data Source and Preparation}

Population data for red fox were obtained from the Living Planet
Database (LPI, 2024), a global repository of vertebrate population time
series derived from scientific publications, online databases, and grey
literature. Each record includes abundance, year, population ID, and
associated metadata (Appendix). Records span 1970--2020, and we analysed
years covering 1970--2000, considering that rabies was eradicated by
2000 (Lojkić et al., 2021).

Based on published literatures (Nyberg et al., 1992; Pastoret \&
Brochier, 1999; Delcourt et al., 2022), countries were classified as
rabies-present (Belarus, Finland) or rabies-absent (Spain, United
Kingdom, United States) (Fig.1). \vspace{1em}

\begin{center}\includegraphics{WWF-Report_files/figure-latex/mapping-1} \end{center}

Fig.1: Map of rabies presence by country where LPD population records
are available (created based on literature review).

\vspace{1em}

Only populations counting live foxes, with ≥5 years of data were
included. Year was scaled so the first observation equaled 0, and
abundance values were multiplied by 10 and rounded to meet Poisson
requirements (see below).

\subsection{Model Choice and Justification}

We fitted a Poisson Generalised Linear Mixed Model (GLMM) with a log
link to model abundance change through time. The Poisson family was
chosen because population abundance is inherently a count data with
positively skewed distribution. Fixed effects were scaled year, rabies
presence, and their interaction to test for differing temporal trends.
Population ID was included as a random effect to account for repeated
measures and differing sampling units among studies.

\subsection{Assumption Checks}

Residual dispersion, normality, and homoscedasticity were evaluated with
DHARMa package, and model singularity was tested using performance
package. All data processing, diagnostics, and visualisation were
conducted in RStudio v4.5.1.

\section{Results}\label{results}

Red fox populations declined by 2.3\% per year in rabies-affected
regions between 1970 and 2000, and populations in rabies-free regions
increased by 1.8\% per year (Fig.2).

\begin{center}\includegraphics{WWF-Report_files/figure-latex/visualisation-1} \end{center}

Fig.2: Population trends over time in rabies-affected and rabies-free
regions. Lines show fitted slopes with 95\% confidence intervals; points
represent raw, scaled abundance. \vspace{1em}

Confidence intervals overlapped early in the period but diverged through
time, and the interaction between year and rabies presence was
significant (\emph{p} \textless{} 0.001; Table.1), evidencing rabies's
impact on long-term population decline.

\pagebreak

\begin{center}
Table.1: Model estimates (standard errors). All estimates are exponentiated.
\end{center}

\begin{center}
 \begin{tabular}{@{\extracolsep{5pt}}lc}  \\[-1.8ex]\hline  \hline \\[-1.8ex]   & \multicolumn{1}{c}{\textit{Dependent variable:}} \\  \cline{2-2}  \\[-1.8ex] & pop \\  \hline \\[-1.8ex]   yearscale & 1.018$^{***}$ \\    & (0.008) \\    & \\   rabiesPresent & 0.399 \\    & (0.215) \\    & \\   yearscale:rabiesPresent & 0.960$^{***}$ \\    & (0.013) \\    & \\   Constant & 104.556$^{**}$ \\    & (36.536) \\    & \\  \hline \\[-1.8ex]  Observations & 300 \\  Log Likelihood & $-$3,721.484 \\  Akaike Inf. Crit. & 7,456.968 \\  Bayesian Inf. Crit. & 7,482.895 \\  \hline  \hline \\[-1.8ex]  \textit{Note:}  & \multicolumn{1}{r}{$^{*}$p$<$0.05; $^{**}$p$<$0.01; $^{***}$p$<$0.001} \\  \end{tabular} \end{center}

\subsection{Prediction Accuracy}

The model explained 71 \% of total variance (conditional R²) and 17 \%
via fixed effects alone (marginal R²). The root-mean-square error
indicated an average prediction deviation of ±9 \%, with greater
uncertainty for rabies-free regions (Fig.2).

\subsection{Assumption}

Model diagnostics using DHARMa found no overdispersion (\emph{p} = 0.23)
or outliers (\emph{p} = 1). We found deviations in residual normality
and homoscedasticity, which were expected due to variation in
measurement units among populations.

\section{Discussion}\label{discussion}

Rabies outbreaks were associated with long-term declines in red fox
populations, as indicated by the significant interaction term. However,
results should be interpreted cautiously due to potential confounding
factors and limitations in temporal and spatial coverage.

\subsection{Representativeness and Confounding Factors}

The model showed strong population dependency, with about 50\% of
variance explained by random effects (conditional--marginal R² gap). In
addition to scatter caused by differing measurement units, limited
temporal coverage per population may bias the model toward short-term
fluctuations.

\vspace{1em}

In rabies-affected regions, mild population-level recoveries observed in
the late 1980s--1990s may correspond to oral vaccination campaigns
(Delcourt et al., 2022).

\vspace{1em}

Population increases in rabies-free regions cannot be explained by
rabies absence alone. These may reflect short-term boosts linked to
urban expansion, where waste and crops provide food resources (Scott et
al., 2014; Jackowiak et al., 2021). Roadside observations may amplify
this effect. In the United States, rabies outbreaks among skunks and
raccoons during the 1970s--1990s may have reduced interspecific
competition, indirectly benefiting foxes (Ma et al., 2020).

\vspace{1em}

Although including population ID as a random slope and intercept helps
address these biases, cautious interpretation is needed.

\vspace{1em}

Spatially, denser populations may be more susceptible to rabies spread,
which depends on the intensity of inter and intraspecific interactions.
Most samples were collected from accessible areas such as national parks
or roadsides. These areas have easier access to anthropogenic food
supply, likely representing high-density ``rabies-vulnerable''
populations. Thus, our findings may not fully capture dynamics in other
ecological communities.

\section{Conservation
Recommendations}\label{conservation-recommendations}

Red fox populations declined where rabies occurred, which makes rabies
monitoring and vaccination crucial to sustain populations. Other factors
such as urban expansion and interspecific competition may be masking our
interpretation. More work on these is needed to identify regional
actions to sustain and control for red fox populations.

\section{Reflection on AI Use}\label{reflection-on-ai-use}

I used ChatGPT to resolve minor coding errors that were unclear from
package documentation and to clarify the logic behind model choices.
While AI provided mathematically accurate explanations, I made my own
decisions based on ecological relevance for choosing a model. This
helped ensure that I made scientific decision on analysis, not just
statistically correct.

\vspace{1em}

I used Perplexity to locate relevant literature, but I found that AI
summaries often lacked critical interpretation. Therefore, I reviewed
the original papers to build proper evidence-based arguments.

\vspace{1em}

When learning about linear and mixed effects models, AI explanations
helped me unpack technical terminology and understand the structure of
the models before consulting formal documentation. This stepwise
approach made reading academic and software references less overwhelming
and improved my confidence in interpreting model outputs.

\section{References}\label{references}

Delcourt, J. et al.~(2022) `Fox Vulpes vulpes population trends in
Western Europe during and after the eradication of rabies', Mammal
Review, 52(3), pp.~343--359. Available at:
\url{https://doi.org/10.1111/mam.12289}. \vspace{0.5em} Jackowiak, M. et
al.~(2021) `Colonization of Warsaw by the red fox Vulpes vulpes in the
years 1976--2019', Scientific Reports, 11, p.~13931. Available at:
\url{https://doi.org/10.1038/s41598-021-92844-2}. \vspace{0.5em} Lojkić,
I. et al.~(2021) `Current Status of Rabies and Its Eradication in
Eastern and Southeastern Europe', Pathogens, 10(6), p.~742. Available
at: \url{https://doi.org/10.3390/pathogens10060742}. \vspace{0.5em} Ma,
X. et al.~(2020) `Public Veterinary Medicine: Public Health: Rabies
surveillance in the United States during 2018'. Available at:
\url{https://doi.org/10.2460/javma.256.2.195}. \vspace{0.5em} LPI (2024)
`Living Planet Index database'. 2024. \textless{}
www.livingplanetindex.org/\textgreater{} \vspace{0.5em} Nyberg, M. et
al.~(1992) `An epidemic of sylvatic rabies in Finland--descriptive
epidemiology and results of oral vaccination', Acta Veterinaria
Scandinavica, 33(1), pp.~43--57. Available at:
\url{https://doi.org/10.1186/BF03546935}. \vspace{0.5em} Pastoret, P.P.
and Brochier, B. (1999) `Epidemiology and control of fox rabies in
Europe', Vaccine, 17(13), pp.~1750--1754. Available at:
\url{https://doi.org/10.1016/S0264-410X(98)00446-0}. \vspace{0.5em}
Scott, D.M. et al.~(2014) `Changes in the Distribution of Red Foxes
(Vulpes vulpes) in Urban Areas in Great Britain: Findings and
Limitations of a Media-Driven Nationwide Survey', PLoS ONE, 9(6),
p.~e99059. Available at:
\url{https://doi.org/10.1371/journal.pone.0099059}.

\section{Appendix}\label{appendix}

Git hub Repository \textless{} Access here:
\url{https://github.com/EdDataScienceEES/2-wwf-report-shiirumini-ms}\textgreater{}


\end{document}
